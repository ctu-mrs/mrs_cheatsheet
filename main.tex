\documentclass[a4paper,landscape]{article}

%%{ packages

\usepackage[landscape]{geometry}
\usepackage{url}
\usepackage{multicol}
\usepackage{amsmath}
\usepackage{amsfonts}
\usepackage{tikz}
\usepackage{amsmath,amssymb}
\usepackage{colortbl}
\usepackage{xcolor}
\usepackage{mathtools}
\usepackage{amsmath,amssymb}
\usepackage{listings}
\usepackage{pdfpages}
\lstset{
  basicstyle=\ttfamily,
  frame=single
}
\usepackage{enumitem}
\usepackage[english]{babel}
\usepackage[utf8]{inputenc}
\usetikzlibrary{decorations.pathmorphing}

%%}

\title{MRS lab ROS platform Cheat Sheet}

\advance\topmargin-.8in
\advance\textheight3in
\advance\textwidth3in
\advance\oddsidemargin-1.5in
\advance\evensidemargin-1.5in
\parindent0pt
\parskip2pt

\newcommand{\hr}{\centerline{\rule{3.5in}{1pt}}}

%\colorbox[HTML]{e4e4e4}{\makebox[\textwidth-2\fboxsep][l]{texto}

\begin{document}

\begin{center}{\huge{\textbf{MRS lab ROS platform Cheat Sheet}}}\\
  {\large by Tomas Baca @ Multi-robot Systems (MRS), v1.1.0}
\end{center}

\begin{multicols*}{2}

  \tikzstyle{mybox} = [draw=black, fill=white, very thick,
  rectangle, rounded corners, inner sep=10pt, inner ysep=10pt]
  \tikzstyle{fancytitle} =[fill=black, text=white, font=\bfseries]

  %%{ TERMINAL BASICS

  \begin{tikzpicture}
    \node [mybox] (box){%
        \begin{minipage}{0.47\textwidth}
          \begin{center}
            \scriptsize{
              \textbf{Hitting $\langle\mathrm{Tab}\rangle$ autocompletes commands, filenames, etc.}
              \begin{tabular}{lp{6.0cm} r}
                \hline
                New terminal & \texttt{Ctrl+Alt+t} \\ \hline
                \textbf{Need help} & append \texttt{--help} after command\\
                \textbf{Need more help!} & \texttt{:\$ man [command]} \\ \hline
                Change directory & \texttt{:\$ cd [path]} \\ \hline
                Path symbolic links & $\texttt{.}$ -- current directory \\
                                    & $\texttt{..}$ -- previous directory \\
                                    & $\texttt{$\sim$}$ -- home directory (also \texttt{\$HOME})\\
                                    & $\texttt{/}$ -- root directory \\ \hline
                create a file & \texttt{:\$ touch [path]} \\
                remove a file & \texttt{:\$ rm [path]} \\
                move (also rename) a file & \texttt{:\$ mv [from] [to]} \\
                copy a file & \texttt{:\$ cp [from] [to]} \\
                print a file & \texttt{:\$ cat [path]} \\
                edit a file & \texttt{:\$ vim [path]}, \texttt{:\$ nano [path]} \\ \hline
                set a variable & \texttt{:\$ VARIABLE="dog", VARIABLE=3.0} \\
                print a variable & \texttt{:\$ echo "the content is: \$VARIABLE"} \\ \hline
                run a script or executable& \texttt{:\$ ./script.sh, ./program} \\ \hline
                output redirection & $\texttt{>}$ -- to a file (rewrite) \\
                                   & $\texttt{>>}$ -- to a file (append) \\
                                   & $\texttt{|}$ -- pipe to another command \\
                redirect to /dev/null & $\texttt{> /dev/null 2>\&1}$ \\ \hline
              \end{tabular}
            }

            \vspace{0.2em}
            Would You Like to Know More? \url{http://google.com}
            \vspace{-1em}
          \end{center}
        \end{minipage}
      };
    \node[fancytitle, right=10pt] at (box.north west) {Ubuntu terminal - GNU/Linux basics};
  \end{tikzpicture}

  %%}

  %%{ TMUX

  \begin{tikzpicture}
    \node [mybox] (box){%
        \begin{minipage}{0.47\textwidth}
          \begin{center}
            \scriptsize{
              \begin{tabular}{lp{4.0cm} r}
                \hline
                Run tmux & \texttt{:\$ tmux} \\
                List all sessions & \texttt{:\$ tmux ls} \\
                Attach to a session & \texttt{:\$ tmux a -t [session name]} \\
                \hline
                New window (tab) & \texttt{Ctrl+t}\\
                New horizontal split & \texttt{Ctrl+9}\\
                New vertical split & \texttt{Ctrl+0}\\
                Moving through windows (tabs) & \texttt{Shift+$\rightarrow$}, \texttt{Shift+$\leftarrow$}  \\
                Moving through panes (splits) & \texttt{Alt+$\rightarrow$}, \texttt{Alt+$\leftarrow$}, \texttt{Alt+$\uparrow$}, \texttt{Alt+$\downarrow$} \\
                \hline
                \textbf{prefix} & \texttt{Ctrl+a}\\
                \hline
                Killing window & \textbf{prefix} \texttt{x}, \texttt{:\$ exit}, \texttt{:\$ :q}\\
                Killing session & \textbf{prefix} \texttt{k}\\
                Detach from session & \textbf{prefix} \texttt{d}\\
                Enter vim mode (scrolling, copying) & \texttt{F2}, \textbf{prefix} \texttt{$[$}\\
                \hline
              \end{tabular}
            }

            \vspace{0.2em}
            Would You Like to Know More? \url{https://github.com/klaxalk/linux-setup/wiki/tmux}
            \vspace{-1em}
          \end{center}
        \end{minipage}
      };
    \node[fancytitle, right=10pt] at (box.north west) {TMUX - Terminal multiplexer};
  \end{tikzpicture}

  %%}

  %%{ VIM

  \begin{tikzpicture}
    \node [mybox] (box){%
        \begin{minipage}{0.47\textwidth}
          \scriptsize{
            Vim is not a joke. Although you might not know how to exit it (yet), it is a very powerful tool.
            Our vim is filled with features, including code snippets, code completion (ROS aware), code formatting, syntax highlighting and tmux integration.
            Its control is completely mouse-less and it is fully usable over ssh, which makes it great for remote editing on a drone.
            Moreover, its modal editing paradigm is very intuitive.
            Lastly, when you learn how to control vim, you also learn to control other tools such as \emph{Linux manual pages}, \emph{ranger}, \emph{less} and much more.
            Even gmail uses vim-like controls natively.
            Run \texttt{:\$ vimtutor} to start learning vim using an interactive ``file tutorial''.
            Here are some simple commands:
            \begin{center}
              \begin{tabular}{lp{1cm} r lp{1cm} r}
                \hline
                switch to insert mode & \texttt{i} & jump a word/Word forwards & \texttt{w}/\texttt{W} \\
                return to normal mode & \texttt{ESC} & jump a word/Word backwards & \texttt{b}/\texttt{B} \\
                cut a line to clipboard & \texttt{dd} & change current word/Word & \texttt{ciw}/\texttt{ciW} \\
                paste a clipboard & \texttt{p} & delete 3 lines down & \texttt{3dj} \\
                \hline
                open a command line & \texttt{:} & substitute \emph{dog} for \emph{cat} & \texttt{:\%s/dog/cat/g} \\
                save & \texttt{:w} & move cursor left/down/up/right & h/j/k/l\\
                quit & \texttt{:q} & delete every line containing \emph{dog} & \texttt{:\%g/dog/normal dd} \\
                \hline
              \end{tabular}

              \vspace{0.2em}
              Would You Like to Know More? \url{https://www.tutorialspoint.com/vim/}
              \vspace{-1em}
            \end{center}
          }
        \end{minipage}
      };
    \node[fancytitle, right=10pt] at (box.north west) {Vim -- a modern modular text processor};
  \end{tikzpicture}

  %%}

  %%{ GIT

  \begin{tikzpicture}
    \node [mybox] (box){%
        \begin{minipage}{0.47\textwidth}
          \scriptsize{Git is a distributed version control system.
            Repositories are equal, some are just used as a ``server'' (called \textbf{remote}).
            Git uses branches to isolate ongoing work on the same project.
            Branches can be merged to combine the work back into a single piece.
            Changes in the files should be \textbf{commited}.
          Only commit ``runnable'' code.}
          \begin{center}
            \scriptsize{
              \begin{tabular}{lp{8.0cm} r}
                \hline
                Cloning a repository & over ssh \texttt{:\$ git clone git@mrs.felk.cvut.cz:uav/uav\_core} \\
                                     & over https \texttt{:\$ git clone https://github.com/klaxalk/linux-setup} \\
                                     \hline
                Update origin state & \texttt{:\$ git fetch} \\
                Update current branch from remote & \texttt{:\$ git pull} \\
                Update current branch to remote & \texttt{:\$ git push} \\
                Commit ``patch'' -- interactive  & \texttt{:\$ git commit -p} \\
                Add files for commit & \texttt{:\$ git add [file]} \\
                Commit changes & \texttt{:\$ git commit -m "commit message"} \\
                \hline
                Checkout a branch & \texttt{:\$ git checkout [branch name]} \\
                Create a branch & \texttt{:\$ git checkout -b [branch name]} \\
                \hline
                unstage the file & \texttt{:\$ git reset [file name]} \\
                undo all uncommited changes & \texttt{:\$ git reset --hard} \\
                remove all new unstaged files & \texttt{:\$ git clean -fd} \\
                \hline
                Merge a branch & \texttt{:\$ git merge [branch name]} \\
                Rebase on a branch & \texttt{:\$ git rebase [branch name]} \\
                \hline
                refactor branch history & \texttt{:\$ git filter-branch [lot of args]} \\
                \hline
                show status & \texttt{:\$ git status} \\
                show log & \texttt{:\$ git log} \\
                show better log & \texttt{:\$ glog} \\
                % & -- is an alias for \texttt{:\$ git log --graph --abbrev-commit --date=relative --pretty=format:'\%Cred\%h\%Creset -\%C(yellow)\%d\%Creset \%s \%Cgreen(\%cr) \%C(bold blue)<\%an>\%Creset'} \\
                  & -- is an alias for \texttt{:\$ git log} with more arguments\\
                show super forest log & \texttt{:\$ flog} \\
                                      & -- uses \texttt{~/.scripts/git-forest.sh} \\
                                      \hline
              \end{tabular}

              \vspace{0.2em}
              Would You Like to Know More? \url{https://try.github.io/}
              \vspace{-1em}
            }
          \end{center}
        \end{minipage}
      };
    \node[fancytitle, right=10pt] at (box.north west) {Git version control system};
  \end{tikzpicture}

  %%}

  %%{ .bashrc

  \begin{tikzpicture}
    \node [mybox] (box){%
        \begin{minipage}{0.47\textwidth}
          \scriptsize{
            When a new terminal is opened and an instance of bash is launched, the \texttt{$\sim$/.bashrc} file is \emph{sourced} (executed while its leftover variables, functions and aliases stay in the context).
            We use \texttt{.bashrc} heavily for setting context for ROS and our development environment.
            \texttt{.bashrc} sources ROS setup scripts, which are also generated by each workspace.
            If you change this file, source it (or open a new terminal) to activate the changes: \texttt{:\$ source $\sim$/.bashrc} or just \texttt{:\$ sb}.
            Here is an example of what should not be missing in the bottom of a healthy \texttt{.bashrc} file:
            \vspace{-1.5em}
            \begin{center}
              \begin{lstlisting}
        source /opt/ros/melodic/setup.bash
        source /usr/share/gazebo/setup.sh

        source ~/workspace/devel/setup.bash
        # source ~/other_workspace/devel/setup.bash

        export ROS_WORKSPACES="~/mrs_workspace ~/workspace"

        export GIT_PATH=$HOME/git

        export RUN_TMUX=true

        # VARIABLES TO CONFIGURE THE MRS ROS PIPELINE
        export UAV_NAME="uav1"
        ...
        export MRS_STATUS="readme"

        source $GIT_PATH/uav_core/miscellaneous/shell_additions/shell_additions.sh

        source $GIT_PATH/linux-setup/appconfig/bash/dotbashrc
              \end{lstlisting}
              \vspace{-1.5em}
            \end{center}
          }
        \end{minipage}

      };
    \node[fancytitle, right=10pt] at (box.north west) {.bashrc -- Bash configuration};
  \end{tikzpicture}

  %%}

  %%{ ROS IN LINUX TERMINAL

  \begin{tikzpicture}
    \node [mybox] (box){%
        \begin{minipage}{0.47\textwidth}
          \begin{center}
            \scriptsize{
              \textbf{Please}, visit \url{http://wiki.ros.org/ROS/Tutorials} before starting work on a bigger project.\\
              \textbf{Use $\langle\mathrm{Tab}\rangle$ to complete commands, topic names, message types and pre-fill message contents.}

              \begin{tabular}{lp{6.0cm} r}
                \hline
                Getting help & append \texttt{--help} after any following command\\
                \hline
                Listing all ROS nodes & \texttt{:\$ rosnode list} \\
                Listing all ROS topics & \texttt{:\$ rostopic list} \\
                Listing all ROS services & \texttt{:\$ rosservice list} \\
                Listing all ROS params & \texttt{:\$ rosparam list} \\
                \hline
                Running a ROS binary & \texttt{:\$ rosrun package\_name binary\_name} \\
                Running a launch file & \texttt{:\$ roslaunch package\_name launch\_file.launch} \\
                \hline
                Showing a node info & \texttt{:\$ rosnode info /node/path} \\
                Showing a topic info & \texttt{:\$ rostopic info /topic/path} \\
                Showing a service info & \texttt{:\$ rosservice info /service/path} \\
                \hline
                Showing a topic type & \texttt{:\$ rostopic type /topic/path} \\
                Showing a service type & \texttt{:\$ rosservice type /topic/path} \\
                \hline
                Showing a message type structure & \texttt{:\$ rosmsg show [msg type]} \\
                Showing a service type structure & \texttt{:\$ rossrv show [srv type]} \\
                \hline
                Showing topic messages & \texttt{:\$ rostopic echo /topic/path} \\
                Showing a param value & \texttt{:\$ rosparam get /parm/path} \\
                \hline
                Calling a service & \texttt{:\$ rosservice call /service/path [args]} \\
                Publishing on a topic & \texttt{:\$ rostopic pub /topic/path [args]} \\
                Setting a param value & \texttt{:\$ rosparam set /parm/path [args]} \\
                \hline
              \end{tabular}
            }

            \vspace{0.2em}
            Would You Like to Know More? \url{http://wiki.ros.org/ROS/CommandLineTools}
            \vspace{-1em}
          \end{center}
        \end{minipage}
      };
    \node[fancytitle, right=10pt] at (box.north west) {ROS in Linux terminal};
  \end{tikzpicture}

  %%}

  %%{ ROS WORKSPACE STRUCTURE

  \begin{tikzpicture}
    \node [mybox] (box){%
        \begin{minipage}{0.47\textwidth}
          \subsubsection*{MRS lab main workspace}
          \begin{center}
            \scriptsize{
              \begin{tabular}{lp{6.0cm} r}
                \hline
                path & \texttt{$\sim$/mrs\_workspace/}\\
                contains & \texttt{src/uav\_core/} -- core MRS repository\\
                         & \texttt{src/uav\_modules/} -- modules MRS repository\\
                         \hline
              \end{tabular}
            }
          \end{center}
          \subsubsection*{MRS lab student workspace}
          \begin{center}
            \scriptsize{
              \begin{tabular}{lp{6.0cm} r}
                \hline
                path & \texttt{$\sim$/workspace}\\
                contains & \texttt{example\_packages/}\\
                         & -- \texttt{waypoint\_flier} -- general example\\
                         & -- \texttt{vision\_example} -- computer vision template\\
                         \hline
              \end{tabular}
            }
          \end{center}
          \subsubsection*{General ROS package structure}
          \begin{center}
            \scriptsize{
              \begin{tabular}{lp{8.0cm} r}
                \hline
                \texttt{build} & generated makefiles and support files\\
                               & \textbf{do not modify}\\
                \texttt{devel} & compiled binaries, libraries and installed headers\\
                               & \textbf{do not modify}\\
                \texttt{src} & package source codes\\
                             & \textbf{place your stuff in here}\\
                             \hline
              \end{tabular}
            }

            \scriptsize{
              \vspace{0.2em}
              Would You Like to Know More? \url{https://mrs.felk.cvut.cz/gitlab/uav/uav_core/wikis/file_structure}\\
              Even more? \url{https://mrs.felk.cvut.cz/gitlab/uav/uav_core/wikis/repositories_structure}
              \vspace{-1em}
            }
          \end{center}
        \end{minipage}
      };
    \node[fancytitle, right=10pt] at (box.north west) {ROS workspace structure};
  \end{tikzpicture}

  %%}

  %%{ NAVIGATING AND COMPILING ROS WORKSPACE

  \begin{tikzpicture}
    \node [mybox] (box){%
        \begin{minipage}{0.47\textwidth}
          \begin{center}
            \scriptsize{
              \begin{tabular}{lp{5cm} r}
                \hline
                go to a package & \texttt{:\$ roscd [package name]}\\
                \hline
                compile the whole workspace & \texttt{:\$ catkin build}\\
                compile a particular package & \texttt{:\$ catkin build [package name]}\\
                compile current package & \texttt{:\$ catkin bt}\\
                clean the whole workspace & \texttt{:\$ catkin clean}\\
                clean a particular package & \texttt{:\$ catkin clean [package name]}\\
                \hline
                show workspace config & \texttt{:\$ catkin config}\\
                show compilation profiles & \texttt{:\$ catkin profile list}\\
                set a compilation profile & \texttt{:\$ catkin profile set [profile name]}\\
                \hline
                create a new workspace & \texttt{:\$ catkin init}\\
                set workspace extending & \texttt{:\$ catkin config --extend [path]}\\
                \hline
              \end{tabular}

              \vspace{0.2em}
              Would You Like to Know More? \url{https://catkin-tools.readthedocs.io/en/latest/}
              \vspace{-1em}
            }
          \end{center}
        \end{minipage}
      };
    \node[fancytitle, right=10pt] at (box.north west) {Navigating and compiling ROS workspace};
  \end{tikzpicture}

  %%}

  %%{ ROS PACKAGE STRUCTURE

  \begin{tikzpicture}
    \node [mybox] (box){%
        \begin{minipage}{0.47\textwidth}
          \scriptsize {Some of the following items might be missing, depending on the package use case.}
          \begin{center}
            \scriptsize{
              \begin{tabular}{lp{5.5cm} r}
                \hline
                \texttt{package.xml} & manifest, dependencies and plugins\\
                \texttt{CMakeLists.txt} & description of compilation procedure\\
                \texttt{src/} & \verb!C! and \verb!C++! source codes\\
                \texttt{include/} & \verb!C! and \verb!C++! headers\\
                \texttt{scripts/} & Python and bash scripts\\
                \texttt{config/} & yaml config files\\
                \texttt{cfg/} & dynamic reconfigure scripts\\
                \texttt{launch/} & ROS launch files\\
                \hline
              \end{tabular}
            }

            \vspace{0.2em}
            Would You Like to Know More? \url{http://wiki.ros.org/Packages}
            \vspace{-1em}
          \end{center}
        \end{minipage}
      };
    \node[fancytitle, right=10pt] at (box.north west) {ROS package structure};
  \end{tikzpicture}

  %%}

  %%{ ROS VISUALIZATION TOOLS

  \begin{tikzpicture}
    \node [mybox] (box){%
        \begin{minipage}{0.47\textwidth}
          \begin{center}
            \scriptsize{
              \begin{tabular}{lp{6cm} r}
                \hline
                Rviz & 3-D visualization of data and models \\
                     & \texttt{:\$ rviz} \\
                     & \texttt{:\$ roslaunch mrs\_testing rviz\_uav1.launch} \\ \hline
                Rqt plot & simple and lightweight plotting \\
                         & \texttt{:\$ rqt\_plot} \\ \hline
                Rqt bag & visualizing contents of a rosbag \\
                        & \texttt{:\$ rqt\_bag} \\ \hline
                Plot juggler & complex and powerful plotting \\
                             & \texttt{:\$ rosrun plotjuggler PlotJuggler} \\ \hline
                Rqt reconfigure & online parameter setting \\
                                & \texttt{:\$ rosrun rqt\_reconfigure rqt\_reconfigure} \\ \hline
                Rqt image view & camera images visualization \\
                               & \texttt{:\$ rqt\_image\_view} \\ \hline
                Gazebo client & Gazebo GUI \\
                              & \texttt{:\$ gzclient} \\ \hline
                rqt & Integrates most of the \textit{rqt\_} tools \\
                    & \texttt{:\$ rqt} \\ \hline
              \end{tabular}
            }

            \vspace{0.2em}
            Would You Like to Know More? \url{http://wiki.ros.org/Tools}
            \vspace{-1em}
          \end{center}
        \end{minipage}
      };
    \node[fancytitle, right=10pt] at (box.north west) {ROS visualization tools};
  \end{tikzpicture}

  %%}

  %%{ USEFUL UAV ROS TOPICS AND SERVICES

  \begin{tikzpicture}
    \node [mybox] (box){%
        \begin{minipage}{0.47\textwidth}
          \scriptsize{
            Following ROS services and topics allow for controlling the UAV from terminal.
            Each address contains a particular name of the UAV.
          }
          \subsubsection*{Informative topics (subscribe to know stuff)}
          \begin{center}
            \scriptsize{
              \begin{tabular}{lp{6cm} r}
                \hline
                state estimate (rviz-able) & \texttt{/uav1/odometry/odom\_main}\\
                control reference (rviz-able) & \texttt{/uav1/control\_manager/cmd\_odom}\\
                control reference (full-state) & \texttt{/uav1/control\_manager/position\_cmd}\\
                control manager diagnostics & \texttt{/uav1/control\_manager/diagnostics}\\
                \hline
              \end{tabular}
            }
          \end{center}
          \subsubsection*{Control Services/Topics (call or publish to influence stuff)}
          \begin{center}
            \scriptsize{
              The addresses for \emph{reference}, \emph{trajectory\_reference} are the same for both the topic and the service.
              \vspace{0.2em}
              \begin{tabular}{lp{8cm} r}
                \hline
                position+heading goal & \texttt{/uav1/control\_manager/reference}\\
                \hline
                takeoff & \texttt{/uav1/uav\_manager/takeoff}\\
                land & \texttt{/uav1/uav\_manager/land}\\
                land home & \texttt{/uav1/uav\_manager/land\_home}\\
                hover & \texttt{/uav1/uav\_manager/hover}\\
                \hline
                switch controller & \texttt{/uav1/control\_manager/switch\_controller [Controller]}\\
                switch tracker & \texttt{/uav1/control\_manager/switch\_tracker [Tracker]}\\
                set tracker constraints & \texttt{/uav1/constraint\_manager/set\_constraints [Constraints]}\\
                set SO(3) controller gains & \texttt{/uav1/gain\_manager/set\_gains [Gains]}\\
                \hline
                load trajectory & \texttt{/uav1/control\_manager/trajectory\_reference}\\
                trajectory goto start & \texttt{/uav1/control\_manager/goto\_trajectory\_start}\\
                trajectory start tracking & \texttt{/uav1/control\_manager/start\_trajectory\_tracking}\\
                \hline
              \end{tabular}

              \vspace{0.2em}
              \tiny Would You Like to Know More? \url{https://mrs.felk.cvut.cz/gitlab/uav/uav_core/wikis/commanding_the_drone}
              \vspace{-1em}
            }
          \end{center}
        \end{minipage}
      };
    \node[fancytitle, right=10pt] at (box.north west) {Useful UAV ROS topics and services};
  \end{tikzpicture}

  %%}

  \clearpage

  %%{ SSH keys

  \begin{tikzpicture}
    \node [mybox] (box){%
        \begin{minipage}{0.47\textwidth}
          \scriptsize{
            \begin{itemize}
              \setlength\itemsep{0.0em}
            \item Generate your SSH key by: \texttt{:\$ ssh-keygen -t rsa -b 4096 -C "your\_email@example.com"}.
            \item The keys are stored in \texttt{$\sim$/.ssh}.
            \item Show the content of the public key by: \texttt{:\$ cat $\sim$/.ssh/id\_rsa.pub} and copy it to Github or Gitlab.
            \item Copy your public key over ssh to another machine by: \texttt{:\$ ssh-copy-id user@machine}.
            \item Entries in the \texttt{$\sim$/.ssh/config} allow connecting to a machine via alias while using an ssh key:
          \end{itemize}
          \vspace{-2.8em}
          \begin{center}
            \begin{lstlisting}
                        host mrs
                             hostname mrs.felk.cvut.cz
                             user git
                             identityfile ~/.ssh/id_rsa
            \end{lstlisting}
          \end{center}
          \vspace{-3.0em}
        }
      \end{minipage}
    };
  \node[fancytitle, right=10pt] at (box.north west) {SSH keys};
\end{tikzpicture}

%%}

%%{ UAV SPAWNING

\begin{tikzpicture}
  \node [mybox] (box){%
      \begin{minipage}{0.47\textwidth}
        \scriptsize{
          We use the \texttt{:\$ rosrun mrs\_simulation spawn} command to dynamically load a UAV into the Gazebo/ROS simulator.
          Various arguments can be used to influence the type of the drone, its sensors, its starting location and additional onboard hardware.
          Type \texttt{:\$ rosrun mrs\_simulation spawn --help} to see the complete list, here are some notable examples:
          \begin{center}
            \begin{tabular}{lp{5cm} r}
              \hline
              start and stop the onboard firmware automatically & \texttt{--run --delete} \\
              use initial position from a CSV file (id, x, y, z, heading) & \texttt{--file [file]} \\
              \hline
              selecting UAV type (f450, f550, t650) & \texttt{--f450}, \texttt{--f550}, \texttt{--t650} \\
              list available sensors for the selected UAV type & \texttt{--available-sensors} \\
              \hline
              add down-facing rangefinder & \texttt{--enable-rangefinder} \\
              add front-facing camera & \texttt{--enable-bluefox-front} \\
              add front-facing RealSense & \texttt{--enable-realsense-front} \\
              add 2-D rangefinder & \texttt{--enable-rplidar} \\
              add 3-D rangefinder & \texttt{--enable-velodyne} \\
              \hline
              add UV camera for UVDAR & \texttt{--enable-uv-camera} \\
              add UV leds for UVDAR & \texttt{--enable-uv-leds} \\
              set UV led frequencies & \texttt{--led-frequencies [L] [R]} \\
              \hline
              add super long pendulum & \texttt{--enable-pendulum} \\
              add ball holder & \texttt{--enable-ball-holder} \\
              \hline
            \end{tabular}
          \end{center}
          A typical simulation spawning looks like:\\
          \texttt{:\$ rosrun mrs\_simulation spawn 1 --f450 --run --delete --enable-rangefinder}
          \vspace{-1em}
        }
      \end{minipage}
    };
  \node[fancytitle, right=10pt] at (box.north west) {Spawning a UAV in Gazebo simulator};
\end{tikzpicture}

%%}

%%{ ROS on a remote machine

\begin{tikzpicture}
  \node [mybox] (box){%
      \begin{minipage}{0.47\textwidth}
        \begin{center}
          \scriptsize{
            \begin{itemize}
              \setlength\itemsep{0.0em}
            \item Add you \textbf{local} machine to the \textbf{remote} machine's \texttt{/etc/hosts} and vice versa.
            \item Make sure the machines can ping each other using their hostname.
            \item Add \texttt{export ROS\_MASTER\_URI=http://localhost:11311} to the \textbf{remote}'s .bashrc.
            \item Add \texttt{export ROS\_MASTER\_URI=http://hostname:11311} to the \textbf{local}'s .bashrc, where \textbf{hostname} is the \textbf{remote}'s hostname.
            \item Run roscore only on the \textbf{remote} machine.
          \end{itemize}
        }
        \vspace{-0.8em}
      \end{center}
    \end{minipage}
  };
\node[fancytitle, right=10pt] at (box.north west) {ROS on a remote machine};
  \end{tikzpicture}

  %%}

  %%{ THE MATH THAT EVERYBODY NEEDS, BUT NOBODY REMEMBERS

  \begin{tikzpicture}
    \node [mybox] (box){%
        \begin{minipage}{0.47\textwidth}
          \begin{multicols*}{2}
            \begin{minipage}{0.30\textwidth}
              \begin{center}
                \scriptsize{2-D rotational matrix:}
                \vspace{-0.5em}
                \small{
                  $$
                  \mathbf{R}\left(\phi\right) = \begin{bmatrix}
                    \cos\phi & -\sin\phi \\
                    \sin\phi& \cos\phi
                  \end{bmatrix}
                  $$
                }
              \end{center}
            \end{minipage}
            \begin{minipage}{0.65\textwidth}
              \begin{center}
                \scriptsize{
                  Degrees-to-radian conversion table with values of $\sin$ and $\cos$:\\
                  \begin{tabular}{c|ccccccc}
                    \hline
                    deg & 0 & 30 & 45 & 60 & 90 & 120 & 180 \\
                    rad & 0 & 0.523 & 0.785 & 1.047 & 1.57 & 2.09 & 3.14 \\
                    \hline
                    $\sin$ & 0.0 & 0.500 & 0.707 & 0.866 & 1.0 & 0.866 & 0.0\\
                    $\cos$ & 1.0 & 0.866 & 0.707 & 0.500 & 0.0 & -0.50 & -1.0
                  \end{tabular}
                }
              \end{center}
            \end{minipage}
          \end{multicols*}
        \end{minipage}
      };
    \node[fancytitle, right=10pt] at (box.north west) {The math that everybody needs, but nobody remembers};
  \end{tikzpicture}

  %%}

  %%{ QR CODE

  % \begin{minipage}{0.40\textwidth}
  %   \vspace{2.5em}
  %   \begin{left}
  %     \includegraphics[height=3.0cm]{./fig/qr.png}
  %   \end{left}
  % \end{minipage}

  %%}

  \vspace{10em}

  %%{ QUATERNIONS

  \begin{tikzpicture}
    \node [mybox] (box){%
        \begin{minipage}{0.47\textwidth}
          \begin{center}
            \scriptsize{
              ``Complex'' numbers with three imaginary parts: $i$, $j$, $k$ and $\|\cdot\| = 1$.
              \begin{tabular}{lp{8cm} r}
                \hline
                By axis $\left[x, y, z\right]$ and angle $\phi$ & $q = \cos\frac{\phi}{2} + \left(xi + yj + zk\right)\sin\frac{\phi}{2}$ \\
                Component-wise & $q_w = \cos\frac{\phi}{2}$, $q_x = x\sin\frac{\phi}{2}$, $q_y = y\sin\frac{\phi}{2}$, $q_z = z\sin\frac{\phi}{2}$ \\
                Inverse quaternion & $q^{-1} = \cos\frac{-\phi}{2} + \left(xi + yj + zk\right)\sin\frac{-\phi}{2} = \frac{q_w - q_xi - q_yj - q_zk}{q_w^2 + q_x^2 + q_y^2 + q_z^2}$ \\
                Transforming the vector $\left[1, 2, 3\right]$ & $u = 0 + 1i + 2j + 3k$, $v = quq^{-1}$ \\
                \hline
              \end{tabular}
              \\
              \vspace{1em}
              Transforming various representations of rotation using \texttt{mrs\_lib::AttitudeConverter}:
              \vspace{-0.5em}
              \begin{lstlisting}
# every combination is possible
tf2::Quaternion           tf2_quat        = AttitudeConverter(roll, pitch, yaw);
tf2::Matrix3x3            tf2_matrix      = AttitudeConverter(tf2_quat);
geometry_msgs::Quaternion gernion         = AttitudeConverter(tf2_matrix);
Eigen::Quaterniond        eig_quat        = AttitudeConverter(gernion);
Eigen::AngleAxis<double>  eig_angle_axis  = AttitudeConverter(eig_quat);
Eigen::Matrix3d           eig_matrix      = AttitudeConverter(eig_angle_axis);
auto [roll2, pitch2, yaw2]                = AttitudeConverter(eig_matrix);
tie(roll2, pitch2, yaw2)                  = AttitudeConverter(roll2, pitch2, yaw2);
double heading1                           = AttitudeConverter(tf2_quat).getHeading();
              \end{lstlisting}
              Would You Like to Know More? \url{https://eater.net/quaternions} \\
              \vspace{-1em}
            }
          \end{center}
        \end{minipage}
      };
    \node[fancytitle, right=10pt] at (box.north west) {Quaternions (unit quaternions)};
  \end{tikzpicture}

  %%}

  %%{ COMMON ROS HANDLERS IN C++

  \begin{tikzpicture}
    \node [mybox] (box){%
        \begin{minipage}{0.47\textwidth}
          \begin{center}
            \scriptsize{
              \begin{tabular}{lp{10cm} r}
                \hline
                node handler & \texttt{ros::NodeHandle nh = ros::NodeHandle("$\sim$");} \\
                nodelet handler & \texttt{ros::NodeHandle nh = nodelet::Nodelet::getMTPrivateNodeHandle();} \\
                \hline
                subscriber & \texttt{ros::Subscriber subscriber = nh.subscribe("name", 1, callback, this, ros::TransportHints().tcpNoDelay());} \\
                publisher & \texttt{ros::Publisher publisher = nh.advertise<message\_class>("name", 1);} \\
                \hline
                service client & \texttt{ros::ServiceClient client = nh.serviceClient<service\_class>("name");} \\
                service server & \texttt{ros::ServiceServer server = nh.advertiseService("name", callback, this);} \\
                \hline
                timer & \texttt{ros::Timer timer = nh.createTimer(ros::Rate(30), callback, this);} \\
                \hline
              \end{tabular}

              \vspace{0.2em}
              Would You Like to Know More? \url{http://wiki.ros.org/ROS/Tutorials}
              \vspace{-1em}
            }
          \end{center}
        \end{minipage}
      };
    \node[fancytitle, right=10pt] at (box.north west) {Common ROS handlers in \verb!C++!};
  \end{tikzpicture}

  %%}

  %%{ COMMON ROS HANDLERS IN Python

  \begin{tikzpicture}
    \node [mybox] (box){%
        \begin{minipage}{0.47\textwidth}
          \begin{center}
            \scriptsize{
              \begin{tabular}{lp{10cm} r}
                \hline
                node handler & \texttt{rospy.init\_node('node\_name', anonymous=True)} \\
                \hline
                subscriber & \texttt{subscriber = rospy.Subscriber('$\sim$topic\_name', MessageClass, callback, queue\_size=1)} \\
                publisher & \texttt{publisher = rospy.Publisher('$\sim$topic\_name', MessageClass, queue\_size=1)} \\
                \hline
                service client & \texttt{client = rospy.ServiceProxy('$\sim$service\_name', MessageClass)}\\
                service server & \texttt{server = rospy.Service('$\sim$service\_name', MessageClass, callback)} \\
                \hline
                timer & \texttt{timer = rospy.Timer(rospy.Duration(1/30.0), callback)} \\
                \hline
              \end{tabular}

              \vspace{0.2em}
              Would You Like to Know More? \url{http://wiki.ros.org/ROS/Tutorials}
              \vspace{-1em}
            }
          \end{center}
        \end{minipage}
      };
    \node[fancytitle, right=10pt] at (box.north west) {Common ROS handlers in Python};
  \end{tikzpicture}

  %%}

  %%{ COMMON EIGEN OPERATIONS IN \VERB!C++!

  \begin{tikzpicture}
    \node [mybox] (box){%
        \begin{minipage}{0.47\textwidth}
          \begin{center}
            \scriptsize{
              \begin{tabular}{ll|ll}
                Fixed matrix & \texttt{Matrix<double, 3, 3> A;} & element-wise product & \texttt{P.cwiseProduct(Q)} \\
                Dynamic matrix & \texttt{MatrixXd A;} & Norm & \texttt{v.norm()}\\
                Dynamic vector & \texttt{VectorXd v;} & Squred norm & \texttt{v.squaredNorm()} \\
                Zero matrix & \texttt{MatrixXd::Zero(rows, cols)} & Dot product & \texttt{v.dot(u)}\\
                Identity matrix & \texttt{MatrixXd::Identity(n, n)} & Cross product & \texttt{v.cross(v)}\\
                Vector element & \texttt{v(n)} & Solve Ax=b & \texttt{x = A.qr().solve(b);} \\
                Matrix element & \texttt{A(row, column)} & Eigen-decomposition & \texttt{EigenSolver<Matrix3d> eig(A);}\\
                Matrix inversion & \texttt{A.inverse()} & Matrix transposition & \texttt{A.transpose()} \\
                Matrix column & \texttt{A.col(n)} & \tiny\texttt{\#include <Eigen/Dense>} & for everything\\
                no. of rows and cols & \texttt{A.rows()}, \texttt{A.cols()} & \tiny\texttt{\#include <Eigen/Geometry>} & for cross\\
                Sub-matrix & \texttt{A.block(i, j, rows, cols)} & \tiny\texttt{\#include <Eigen/QR>} & for QR decomposition\\
              \hline
              \end{tabular}
              \vspace{0.2em}
              Would You Like to Know More? \url{https://eigen.tuxfamily.org/dox/AsciiQuickReference.txt}
              \vspace{-1em}
            }
          \end{center}
        \end{minipage}
      };
    \node[fancytitle, right=10pt] at (box.north west) {Common Eigen operations in \verb!C++!};
  \end{tikzpicture}

  %%}

  %%{ WOULD YOU LIKE TO KNOW MORE

  % \begin{minipage}{0.40\textwidth}
  %   \vspace{3.7em}
  %   \begin{center}
  %     \begin{multicols*}{2}

  %       \begin{minipage}{0.30\textwidth}
  %         \begin{left}
  %           \includegraphics[height=3.0cm]{./fig/more.png}
  %         \end{left}
  %       \end{minipage}

  %       \begin{minipage}{0.70\textwidth}
  %         \begin{center}
  %           \vspace{1.8em}
  %           \huge \textbf{visit the MRS lab!}\\
  %           \huge \textbf{(and use Google)}\\
  %         \end{center}
  %       \end{minipage}

  %     \end{multicols*}
  %   \end{center}
  % \end{minipage}

  %%}

\end{multicols*}

\end{document}
